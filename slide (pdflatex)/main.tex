\documentclass{beamer}

\usepackage{algorithm,algorithmic}
\usepackage{epsfig}
\usepackage{subfigure}
\usepackage{color,xcolor}
\usepackage{amsmath}
\usepackage{amssymb,amsthm}
\usepackage{times}
\usepackage{algorithmic}
\usepackage{algorithm}
\usepackage{bm}
\usepackage{url}
\usepackage{multirow}
\usepackage{multicol}
\usepackage{subfigure}

% For citations
% \usepackage{mlapa}
\newtheorem{defn}{Definition}
\newtheorem{thm}{Theorem}
\newtheorem{prop}{Proposition}
\newtheorem{remark}{Remark}

\usepackage[english]{babel}
\usefonttheme{professionalfonts}
\mode<presentation>
{
    % more theme, see http://hartwork.org/beamer-theme-matrix/
    \usetheme{CambridgeUS}
    \usecolortheme{orchid}

    \setbeamercovered{transparent}
}


% use \bm as bolded fonts
\usepackage{amsthm,amssymb}
\usepackage{bm}

% mathcal
\newcommand{\Ac}{{\mathcal A}}
\newcommand{\Bc}{{\mathcal B}}
\newcommand{\Cc}{{\mathcal C}}
\newcommand{\Dc}{{\mathcal D}}
\newcommand{\Ec}{{\mathcal E}}
\newcommand{\Fc}{{\mathcal F}}
\newcommand{\Gc}{{\mathcal G}}
\newcommand{\Hc}{{\mathcal H}}
\newcommand{\Ic}{{\mathcal I}}
\newcommand{\Jc}{{\mathcal J}}
\newcommand{\Kc}{{\mathcal K}}
\newcommand{\Mc}{{\mathcal M}}
\newcommand{\Nc}{{\mathcal N}}
\newcommand{\Oc}{{\mathcal O}}
\newcommand{\Pc}{{\mathcal P}}
\newcommand{\Qc}{{\mathcal Q}}
\newcommand{\Rc}{{\mathcal R}}
\newcommand{\Sc}{{\mathcal S}}
\newcommand{\Tc}{{\mathcal T}}
\newcommand{\Uc}{{\mathcal U}}
\newcommand{\Vc}{{\mathcal V}}
\newcommand{\Wc}{{\mathcal W}}
\newcommand{\Xc}{{\mathcal X}}
\newcommand{\Yc}{{\mathcal Y}}
\newcommand{\Zc}{{\mathcal Z}}

% mathbf
\newcommand{\abf}{{\bm a}}
\newcommand{\bbf}{{\bm b}}
\newcommand{\cbf}{{\bm c}}
\newcommand{\dbf}{{\bm b}}
\newcommand{\e}{{\bm e}}
\newcommand{\f}{{\bm f}}
\newcommand{\g}{{\bm g}}
\newcommand{\h}{{\bm h}}
\newcommand{\m}{{\bm m}}
\newcommand{\n}{{\bm n}}
\newcommand{\p}{{\bm p}}
\newcommand{\q}{{\bm q}}
\newcommand{\rbf}{{\bm r}}
\newcommand{\s}{{\bm s}}
\newcommand{\tbf}{{\bm t}}
\newcommand{\ubf}{{\bm u}}
\newcommand{\vbf}{{\bm v}}
\newcommand{\w}{{\bm w}}
\newcommand{\x}{{\bm x}}
\newcommand{\y}{{\bm y}}
\newcommand{\z}{{\bm z}}


\newcommand{\A}{{\bm A}}
\newcommand{\B}{{\bm B}}
\newcommand{\C}{{\bm C}}
\newcommand{\D}{{\bm D}}
\newcommand{\E}{{\bm E}}
\newcommand{\F}{{\bm F}}
\newcommand{\G}{{\bm G}}
\newcommand{\Hbf}{{\bm G}}
\newcommand{\Ibf}{{\bm I}}
\newcommand{\J}{{\bm J}}
\newcommand{\K}{{\bm K}}
\newcommand{\Lbf}{{\bm L}}
\newcommand{\M}{{\bm M}}
\newcommand{\N}{{\bm N}}
\newcommand{\Obf}{{\bm O}}
\newcommand{\Pbf}{{\bm P}}
\newcommand{\Q}{{\bm Q}}
\newcommand{\R}{{\bm Q}}
\newcommand{\Sbf}{{\bm S}}
\newcommand{\Tbf}{{\bm T}}
\newcommand{\U}{{\bm U}}
\newcommand{\V}{{\bm V}}
\newcommand{\W}{{\bm W}}
\newcommand{\X}{{\bm X}}
\newcommand{\Y}{{\bm Y}}
\newcommand{\Z}{{\bm Z}}

% mathbb
\newcommand{\Ab}{{\mathbb A}}
\newcommand{\Bb}{{\mathbb B}}
\newcommand{\Cb}{{\mathbb C}}
\newcommand{\Db}{{\mathbb D}}
\newcommand{\Eb}{{\mathbb E}}
\newcommand{\Fb}{{\mathbb F}}
\newcommand{\Gb}{{\mathbb G}}
\newcommand{\Hb}{{\mathbb H}}
\newcommand{\Ib}{{\mathbb I}}
\newcommand{\Jb}{{\mathbb J}}
\newcommand{\Kb}{{\mathbb K}}
\newcommand{\Lb}{{\mathbb L}}
\newcommand{\Mb}{{\mathbb M}}
\newcommand{\Nb}{{\mathbb N}}
\newcommand{\Ob}{{\mathbb O}}
\newcommand{\Pb}{{\mathbb P}}
\newcommand{\Qb}{{\mathbb Q}}
\newcommand{\Rb}{{\mathbb R}}
\newcommand{\Sb}{{\mathbb S}}
\newcommand{\Tb}{{\mathbb T}}
\newcommand{\Ub}{{\mathbb U}}
\newcommand{\Vb}{{\mathbb V}}
\newcommand{\Wb}{{\mathbb W}}
\newcommand{\Xb}{{\mathbb X}}
\newcommand{\Yb}{{\mathbb Y}}
\newcommand{\Zb}{{\mathbb Z}}

% mathfrak, not commonly used.
\newcommand{\Rfr}{{\mathfrak R}}

% bold greeks (must usepackage bm!)
\newcommand{\balpha}{{\bm\alpha}}
\newcommand{\blambda}{{\bm\lambda}}
\newcommand{\bbeta}{{\bm\beta}}
\newcommand{\bgamma}{{\bm\gamma}}
\newcommand{\btheta}{{\bm\theta}}
\newcommand{\bxi}{{\bm\xi}}
\newcommand{\bphi}{{\bm\phi}}
\newcommand{\bpsi}{{\bm\psi}}
\newcommand{\bfeta}{{\bm\eta}}      % be careful of bold \eta
\newcommand{\bnu}{{\bm\nu}}
\newcommand{\btau}{{\bm\tau}}
\newcommand{\bsigma}{{\bm\sigma}}

% special
\newcommand{\T}{\top}               % transpose notation, distinct with \Tbf
\newcommand{\sto}{\mathrm{s.t.}~}
\newcommand{\prox}{\mathrm{prox}}
\newcommand{\sign}{\mathrm{sign}}
\newcommand{\diag}{\mathrm{diag}}
\newcommand{\conv}{\mathbf{conv}}
\newcommand{\onef}{{\bm 1}}
\newcommand{\dom}{\mathbf{dom}~}
\newcommand{\relint}{\mathbf{relint}~}
\newcommand{\tr}{Tr}
% \newcommand{\tr}{\mathrm{tr}}     % choose what you like 
\newcommand{\intd}{\mathrm{d}}
\newcommand{\const}{\mathrm{const}}

% use \mathbf as bolded fonts
% \usepackage{amsthm,amssymb}
\usepackage{bm}

% mathcal
\newcommand{\Ac}{{\mathcal A}}
\newcommand{\Bc}{{\mathcal B}}
\newcommand{\Cc}{{\mathcal C}}
\newcommand{\Dc}{{\mathcal D}}
\newcommand{\Ec}{{\mathcal E}}
\newcommand{\Fc}{{\mathcal F}}
\newcommand{\Gc}{{\mathcal G}}
\newcommand{\Hc}{{\mathcal H}}
\newcommand{\Ic}{{\mathcal I}}
\newcommand{\Jc}{{\mathcal J}}
\newcommand{\Kc}{{\mathcal K}}
\newcommand{\Mc}{{\mathcal M}}
\newcommand{\Nc}{{\mathcal N}}
\newcommand{\Oc}{{\mathcal O}}
\newcommand{\Pc}{{\mathcal P}}
\newcommand{\Qc}{{\mathcal Q}}
\newcommand{\Rc}{{\mathcal R}}
\newcommand{\Sc}{{\mathcal S}}
\newcommand{\Tc}{{\mathcal T}}
\newcommand{\Uc}{{\mathcal U}}
\newcommand{\Vc}{{\mathcal V}}
\newcommand{\Wc}{{\mathcal W}}
\newcommand{\Xc}{{\mathcal X}}
\newcommand{\Yc}{{\mathcal Y}}
\newcommand{\Zc}{{\mathcal Z}}

% mathbf
\newcommand{\abf}{{\mathbf a}}
\newcommand{\bbf}{{\mathbf b}}
\newcommand{\cbf}{{\mathbf c}}
\newcommand{\dbf}{{\mathbf b}}
\newcommand{\e}{{\mathbf e}}
\newcommand{\f}{{\mathbf f}}
\newcommand{\g}{{\mathbf g}}
\newcommand{\h}{{\mathbf h}}
\newcommand{\m}{{\mathbf m}}
\newcommand{\n}{{\mathbf n}}
\newcommand{\p}{{\mathbf p}}
\newcommand{\q}{{\mathbf q}}
\newcommand{\rbf}{{\mathbf r}}
\newcommand{\s}{{\mathbf s}}
\newcommand{\tbf}{{\mathbf t}}
\newcommand{\ubf}{{\mathbf u}}
\newcommand{\vbf}{{\mathbf v}}
\newcommand{\w}{{\mathbf w}}
\newcommand{\x}{{\mathbf x}}
\newcommand{\y}{{\mathbf y}}
\newcommand{\z}{{\mathbf z}}


\newcommand{\A}{{\mathbf A}}
\newcommand{\B}{{\mathbf B}}
\newcommand{\C}{{\mathbf C}}
\newcommand{\D}{{\mathbf D}}
\newcommand{\E}{{\mathbf E}}
\newcommand{\F}{{\mathbf F}}
\newcommand{\G}{{\mathbf G}}
\newcommand{\Hbf}{{\mathbf G}}
\newcommand{\Ibf}{{\mathbf I}}
\newcommand{\J}{{\mathbf J}}
\newcommand{\K}{{\mathbf K}}
\newcommand{\Lbf}{{\mathbf L}}
\newcommand{\M}{{\mathbf M}}
\newcommand{\N}{{\mathbf N}}
\newcommand{\Obf}{{\mathbf O}}
\newcommand{\Pbf}{{\mathbf P}}
\newcommand{\Q}{{\mathbf Q}}
\newcommand{\R}{{\mathbf Q}}
\newcommand{\Sbf}{{\mathbf S}}
\newcommand{\Tbf}{{\mathbf T}}
\newcommand{\U}{{\mathbf U}}
\newcommand{\V}{{\mathbf V}}
\newcommand{\W}{{\mathbf W}}
\newcommand{\X}{{\mathbf X}}
\newcommand{\Y}{{\mathbf Y}}
\newcommand{\Z}{{\mathbf Z}}

% mathbb
\newcommand{\Ab}{{\mathbb A}}
\newcommand{\Bb}{{\mathbb B}}
\newcommand{\Cb}{{\mathbb C}}
\newcommand{\Db}{{\mathbb D}}
\newcommand{\Eb}{{\mathbb E}}
\newcommand{\Fb}{{\mathbb F}}
\newcommand{\Gb}{{\mathbb G}}
\newcommand{\Hb}{{\mathbb H}}
\newcommand{\Ib}{{\mathbb I}}
\newcommand{\Jb}{{\mathbb J}}
\newcommand{\Kb}{{\mathbb K}}
\newcommand{\Lb}{{\mathbb L}}
\newcommand{\Mb}{{\mathbb M}}
\newcommand{\Nb}{{\mathbb N}}
\newcommand{\Ob}{{\mathbb O}}
\newcommand{\Pb}{{\mathbb P}}
\newcommand{\Qb}{{\mathbb Q}}
\newcommand{\Rb}{{\mathbb R}}
\newcommand{\Sb}{{\mathbb S}}
\newcommand{\Tb}{{\mathbb T}}
\newcommand{\Ub}{{\mathbb U}}
\newcommand{\Vb}{{\mathbb V}}
\newcommand{\Wb}{{\mathbb W}}
\newcommand{\Xb}{{\mathbb X}}
\newcommand{\Yb}{{\mathbb Y}}
\newcommand{\Zb}{{\mathbb Z}}


% mathfrak
\newcommand{\Rfr}{{\mathfrak R}}

% bold greeks (must usepackage bm!)
\newcommand{\balpha}{{\bm\alpha}}
\newcommand{\blambda}{{\bm\lambda}}
\newcommand{\bbeta}{{\bm\beta}}
\newcommand{\bgamma}{{\bm\gamma}}
\newcommand{\btheta}{{\bm\theta}}
\newcommand{\bxi}{{\bm\xi}}
\newcommand{\bphi}{{\bm\phi}}
\newcommand{\bpsi}{{\bm\psi}}
\newcommand{\bfeta}{{\bm\eta}}      % be careful of bold \eta
\newcommand{\bnu}{{\bm\nu}}
\newcommand{\btau}{{\bm\tau}}
\newcommand{\bsigma}{{\bm\sigma}}


\newcommand{\bPhi}{{\bm\Phi}}

% special
\newcommand{\T}{\top}               % transpose notation, distinct with \Tbf
\newcommand{\sto}{\mathrm{s.t.}~}
\newcommand{\prox}{\mathrm{prox}}
\newcommand{\sign}{\mathrm{sign}}
\newcommand{\diag}{\mathrm{diag}}
\newcommand{\conv}{\mathbf{conv}}
\newcommand{\onef}{{\bm 1}}
\newcommand{\dom}{\mathbf{dom}~}
\newcommand{\relint}{\mathbf{relint}~}
\newcommand{\tr}{Tr}
% \newcommand{\tr}{\mathrm{tr}}     % choose what you like 
\newcommand{\intd}{\mathrm{d}}
\newcommand{\const}{\mathrm{const}}

\setbeamertemplate{bibliography item}[text]

\begin{document}
%\begin{frame}
    \title[KKT, prox, projection]{The Fastest $\ell_{\infty,1}$ Prox in the West}
    \author[Apple Zhang]{B\'ejar, Benjam\'in and Dokmani\'c, Ivan and Vidal, Ren\'e\\~\\\small Speaker: Apple Zhang} % 显示作者
    \institute[SZU]{Shenzhen University} % 设置学院机构
    \date{\today}  % 显示日期
\begin{frame}
    \titlepage
\end{frame}

\section[Table of Contents]{}
%%%%%%%%%%%%%%%%%%%%%%%%%%%%%%%%%%%%%%%%
\frame {
    \frametitle{\secname}
    \tableofcontents
}

\AtBeginSection[] {
    \begin{frame}
        \frametitle{Table of Contents}
        \tableofcontents[currentsection]
    \end{frame}
}

% =========================================================================
\section{Our Problem}

% +++++++++++++++++++++++++++++++++++++++++++++++++++++++++++++++++++++++++
\begin{frame}[fragile]
    \frametitle{Sparse regression with $L_{\infty,1}$ constraints}
    Consider the following optimization model \cite{L1infprox}:
    \begin{align}
        \min_{\W}\|\Y-\X\W\|_2^2,\quad\st\|\W\|_{\infty,1}\leq\tau.
    \end{align}
\end{frame}
% +++++++++++++++++++++++++++++++++++++++++++++++++++++++++++++++++++++++++

% +++++++++++++++++++++++++++++++++++++++++++++++++++++++++++++++++++++++++
\begin{frame}[fragile]
    \frametitle{Recall $L_{p,q}$-norm}
    Define $L_{p,q}$-norm
    \begin{equation}
        \|\W\|_{p,q}=\left(\sum_{i=1}^m\|\w^{(i)}\|_p^q\right)^{1/q}
    \end{equation}
    And we usually use $L_{2,1}$-norm:
    \begin{equation}
        \|\W\|_{2,1}=\sum_{i=1}^m\|\w^{(i)}\|_2
    \end{equation}
    In this paper, we will use $L_{\infty,1}$-norm and $L_{1,\infty}$-norm:
    \begin{align}
        &\|\W\|_{\infty,1}=\sum_{i=1}^m\|\w^{(i)}\|_\infty=\sum_{i=1}^m\max_{j=1,2,\cdots,c}|w_{ij}|,\\
        &\|\W\|_{1,\infty}=\max_{i=1,2,\cdots,m}\|\w^{(i)}\|_1.
    \end{align}
\end{frame}
% +++++++++++++++++++++++++++++++++++++++++++++++++++++++++++++++++++++++++


% =========================================================================
\section{Preliminary}

% +++++++++++++++++++++++++++++++++++++++++++++++++++++++++++++++++++++++++
\begin{frame}[fragile]
    \frametitle{KKT conditions for convex problems}
    For the following convex problem, where $f_i(\x)$ are convex functions.
    \begin{align}
        \min_{\x}f_0(\x),\quad\st f_i(\x)\leq 0~(i=1,2,\cdots,m),\quad\A\x=\bbf.
    \end{align}
    We can write Lagrange function:
    \begin{equation*}
        L(\x,\balpha,\bbeta)=f_0(\x)+\sum_{i=1}^m\alpha_if_i(\x)+\bbeta^\T(\bbf-\A\x).
    \end{equation*}
    And its KKT conditions are
    \begin{align}
        &\x^*=\arg\min_{\x}L(\x,\balpha^*,\bbeta^*),\\
        &f_i(\x^*)\leq 0,\quad \A\x^*=\bbf,\quad\alpha_i^*\geq 0,\quad\alpha_i^*f_i(\x^*)=0.
    \end{align}
    Note that if $f_i(\x),~i=0,1,\cdots,m$ are differentiable, then we have
    \begin{equation}
        \nabla_{\x} L=\nabla_{\x} f_0(\x)+\sum_{i=1}^m\alpha_i\nabla_{\x} f_i(\x) -\A^\T\bbeta=\bm0.
    \end{equation}
    \vspace{10pt}
\end{frame}
% +++++++++++++++++++++++++++++++++++++++++++++++++++++++++++++++++++++++++

% +++++++++++++++++++++++++++++++++++++++++++++++++++++++++++++++++++++++++
\begin{frame}[fragile]
    \frametitle{Proximal operator}
    For a convex function $f(\cdot)$, define its proximal operator as
    \begin{equation*}
        \prox_f(\x)=\arg\min_{\z}f(\z)+\frac12\|\z-\x\|_2^2,
    \end{equation*}
    which means minimize $f(x)$ and make the solution is close to $\x$.
\end{frame}
% +++++++++++++++++++++++++++++++++++++++++++++++++++++++++++++++++++++++++

% +++++++++++++++++++++++++++++++++++++++++++++++++++++++++++++++++++++++++
\begin{frame}[fragile]
    \frametitle{Projection: a special case of proximal operator}
    For non-empty convex set $\Cc$, define
    \begin{equation}
        \pi_{\Cc}(\x)=\arg\min_{\z\in\Cc}\frac12\|\z-\x\|_2^2.
    \end{equation}
    Projection is a special proximal operator since
    \begin{equation}
        \pi_{\Cc}(\x)=\prox_{\Ib(\cdot\in\Cc)}(\x)=\arg\min_{\z}\Ib(\z\in\Cc)+\frac12\|\z-\x\|_2^2.
    \end{equation}
    where $\Ib(\cdot\in\Cc)$ is indicator function:
    \begin{equation}
        \Ib(\z\in\Cc)=\begin{cases}
            0,      &\z\in\Cc,\\
            +\infty,&\mathrm{otherwise}.
        \end{cases}
    \end{equation}
\end{frame}
% +++++++++++++++++++++++++++++++++++++++++++++++++++++++++++++++++++++++++

% +++++++++++++++++++++++++++++++++++++++++++++++++++++++++++++++++++++++++
\begin{frame}[fragile]
    \frametitle{Example: projection onto of $L_1$-ball}
    Consider $L_1$-ball projection:
    \begin{equation}
        \pi_{\|\cdot\|_1\leq\lambda}(\x)=\arg\min_{\z}\frac12\|\z-\x\|_2^2,\quad\st \|\z\|_1\leq\lambda.
    \end{equation}
    Lagrange:
    \begin{align}
        L(\z,\alpha)
        &=\frac12\|\z-\x\|_2^2+\alpha(\|\z\|_1-\lambda)\\
        &={\color{red}\left[\frac12\|\z-\x\|_2^2+\alpha\|\z\|_1\right]}-\alpha\lambda.
    \end{align}
    Utilize \textbf{KKT conditions},
    \begin{gather}
        \z=\arg\min_{\z}L(\z,\alpha)=\prox_{\alpha\|\cdot\|_1}(\x),\\
        \alpha\geq 0,\quad\alpha(\|\z\|_1-\lambda)=0.
    \end{gather}
    \vspace{15pt}
\end{frame}
% +++++++++++++++++++++++++++++++++++++++++++++++++++++++++++++++++++++++++

% +++++++++++++++++++++++++++++++++++++++++++++++++++++++++++++++++++++++++
\begin{frame}[fragile]
    \frametitle{Example: projection onto of $L_1$-ball}
    \begin{algorithm}[H]
        \caption{$O(n\log n)$ Algorithm for $\pi_{\|\cdot\|_1\leq\lambda}$}
        \begin{algorithmic}[1]
            \REQUIRE $\x\in\Rb^n,\lambda>0$.
            \ENSURE $\z=\pi_{\|\cdot\|_1\leq\lambda}(\x)$.
            \IF {$\|\x\|_1\leq\lambda$}
                \STATE $\z\leftarrow\x$.
            \ELSE
                \STATE $\ubf\leftarrow\mathrm{sort}(\x,\verb+`descend`+)$,
                \STATE $\rho\leftarrow\max\{j=1,2,\cdots,n~|~ u_j-(\sum_{r=1}^ju_r-\lambda)/j>0\}$,
                \STATE $\alpha\leftarrow(\sum_{r=1}^{\rho}u_i-\lambda)/\rho$,
                \STATE $z_i\leftarrow\sign(x_i)[u_i-\alpha]_+,\quad\forall i=1,2,\cdots,n$.
            \ENDIF
        \end{algorithmic}
    \end{algorithm}
\end{frame}
% +++++++++++++++++++++++++++++++++++++++++++++++++++++++++++++++++++++++++


% =========================================================================
\section{Methodology}

% +++++++++++++++++++++++++++++++++++++++++++++++++++++++++++++++++++++++++
\begin{frame}[fragile]
    \frametitle{Projected gradient descent}
    \begin{align}
        \min_{\W}\|\Y-\X\W\|_2^2,\quad\st\|\W\|_{\infty,1}\leq\tau.
    \end{align}
    Update 
    \begin{align}
        \V&\leftarrow\W-\eta\nabla_{\W}J(\W),\\
        \W&\leftarrow\pi_{\|\cdot\|_{\infty,1}\leq\tau}(\V).
    \end{align}
\end{frame}
% +++++++++++++++++++++++++++++++++++++++++++++++++++++++++++++++++++++++++

% +++++++++++++++++++++++++++++++++++++++++++++++++++++++++++++++++++++++++
\begin{frame}[fragile]
    \frametitle{Projection and proximal operator}
    \begin{lemma}
        For all $\tau>0$,
        \begin{equation}
            \prox_{\tau\|\cdot\|_{1,\infty}}(\V)+\pi_{\|\cdot\|_{\infty,1}\leq\tau}(\V)=\V
        \end{equation}
    \end{lemma}
    Therefore, we should first compute $\prox_{\tau\|\cdot\|_{1,\infty}}(\V)$, i.e.,
    \begin{equation}
        \prox_{\tau\|\cdot\|_{1,\infty}}(\V)=\arg\min_{\W}\tau\|\W\|_{1,\infty}+\frac12\|\W-\V\|_2^2.
    \end{equation}
    \vspace{8pt}
    Just for reminder, the definition of $L_{1,\infty}$ is
    \begin{equation}
        \|\W\|_{1,\infty}=\max_{i=1,2,\cdots,m}\|\w^{(i)}\|_1.
    \end{equation}
\end{frame}
% +++++++++++++++++++++++++++++++++++++++++++++++++++++++++++++++++++++++++

% =========================================================================
\section{Experiments}

% +++++++++++++++++++++++++++++++++++++++++++++++++++++++++++++++++++++++++
\begin{frame}[fragile]
    \frametitle{$L_{\infty,1}$ v.s. $L_{2,1}$}
    \scriptsize
    \begin{table}
        \begin{tabular}{cccc}
            \hline\hline
            Datasets    & RFS ($L_{2,1}$) & $L_{\infty,1}$-regression & SVM\\
            \hline
            DNA         & 91.69$\pm$0.97  & 93.60$\pm$0.59 & 90.51$\pm$1.31\\
            \hline
            Binaryalpha & 55.27$\pm$2.56  & 57.53$\pm$1.37 & 62.81$\pm$1.81\\
            \hline
            USPS        & 88.10$\pm$0.51  & 88.26$\pm$0.38 & 92.10$\pm$0.59\\
            \hline\hline
        \end{tabular}
    \end{table}
\end{frame}
% +++++++++++++++++++++++++++++++++++++++++++++++++++++++++++++++++++++++++

% +++++++++++++++++++++++++++++++++++++++++++++++++++++++++++++++++++++++++
\begin{frame}
    \centering
    \Huge{Thank you!}

    \vspace{20pt}
    \scriptsize
    \bibliographystyle{ieeetr}
    \bibliography{ref}
\end{frame}
% +++++++++++++++++++++++++++++++++++++++++++++++++++++++++++++++++++++++++

\end{document}

% https://web.stanford.edu/~jduchi/projects/DuchiShSiCh08/ProjectOntoL1Ball.m